\documentclass[a4paper]{article}
\usepackage{graphics}
%%\usepackage[english,greek]{babel}
\setlength{\parindent}{0mm}
\setlength{\parskip}{1.8mm}
\newtheorem{definition}{Definition}
\newtheorem{theorem}{Theorem}
\newtheorem{proof}{Proof}
\title{Finite difference scheme for a Kirchoff thin plate with free edges}
%%\date{$28^{th}$ August, 2007}
\author{Daniel Tang}
%%\linespread{1.3}

\begin{document}
%%\selectlanguage{english}
\maketitle
The vibrations in a thin, isotropic, homogeneous plate can be described as
\[
\frac{\partial^2 u}{\partial t^2} = -\kappa^2 \nabla^2 \nabla^2 u
\]
where
\[
\kappa = \frac{EH^2}{12\rho(1-\nu^2)}
\]
where $E$ is the Youngs modulus of the material, $H$ is the thickness of the plate, $\rho$ is its density and $\nu$ is its Poisson ratio.

For free edges, we have the boundary conditions (along a left-hand edge):
\[
\frac{\partial^2u}{\partial x^2} + \nu \frac{\partial^2u}{\partial y^2} = 0
\]
and
\[
\frac{\partial^3u}{\partial x^3} + (2-\nu) \frac{\partial^3u}{\partial xy^2} = 0
\]
while at a corner
\[
\frac{\partial^2u}{\partial xy} = 0
\]
Boundary conditions for other edge orientations can be inferred by symmetry.

Using the notation $\partial_x$, $\partial_{xx}$ etc to mean the second order, central finite difference approximation of $\frac{\partial}{\partial x}$, $\frac{\partial^2}{\partial x^2}$ etc. and $\partial_{x+}$ and $\partial_{x-}$ to mean the second order, forward and backward finite difference respectively, we can express the thin plate equation as a finite difference scheme:
\[
\partial_{tt} u = -\kappa (\partial_{xx} + \partial_{yy})(\partial_{xx} + \partial_{yy})u
\]
The boundary conditions along a left-hand edge can be expressed
\begin{equation}
\partial_{xx}u + \nu \partial_{yy}u = 0
\label{boundary1}
\end{equation}
\begin{equation}
\partial_{x-}(\partial_{xx}u + (2-\nu)\partial_{yy}u) = 0
\label{boundary2}
\end{equation}
and at a bottom-left corner as
\begin{equation}
\partial_{x-y-}u = 0
\label{boundarycorner}
\end{equation}

\subsection{Laplacian finite difference at an edge}
At a left-hand edge, from equation \ref{boundary1} we have
\[
\partial_{xx} = - \nu \partial_{yy} 
\]
so
\begin{equation}
\nabla^2 = \partial_{xx} + \partial_{yy} = (1-\nu)\partial_{yy}
\label{delta}
\end{equation}

Rearranging equation \ref{boundary2} gives
\[
\partial_{x-}(\nabla^2 + (1-\nu)\partial_{yy}) = 0
\]
so, at a left-hand edge
\begin{equation}
\partial_{x-}\nabla^2 = (1-\nu)\partial_{x-yy}
\end{equation}
but
\begin{equation}
\partial_{xx}\nabla^2 = \frac{\partial_{x+}\nabla^2 - \partial_{x-}\nabla^2}{\Delta x} = \frac{\partial_{x+}\nabla^2 - (1-\nu)\partial_{x-yy}}{\Delta x}
\label{xdelta}
\end{equation}
where $\Delta x$ is the grid spacing in the $x$ direction.

Differentiating equation \ref{boundary1} with respect to y gives
\[
\partial_{xxyy} = \frac{\partial_{x+yy} - \partial_{x-yy}}{\Delta x}= -\nu\partial_{yyyy}
\]
so
\[
\partial_{x-yy} = \partial_{x+yy} + \Delta x\nu\partial_{yyyy}
\]
substituting back into equation \ref{xdelta}
\[
\partial_{xx}\nabla^2 = \frac{\partial_{x+}\nabla^2 -(1-\nu)(\partial_{x+yy} + \Delta x\nu\partial_{yyyy})}{\Delta x}
\]
but, differentiating equation \ref{delta} by y gives
\[
\partial_{yy}\nabla^2 = (1-\nu)\partial_{yyyy}
\]
so, at a left-hand edge
\begin{equation}
\nabla^2\nabla^2 = \frac{\partial_{x+}\nabla^2 -(1-\nu)\partial_{x+yy}}{\Delta x} - (1-\nu)^2\partial_{yyyy}
\end{equation}
simplifying gives
\begin{equation}
\nabla^2\nabla^2 = \frac{\partial_{x+}(\partial_{xx} +\nu\partial_{yy})}{\Delta x} - (1-\nu)^2\partial_{yyyy}
\label{DeltaDelta}
\end{equation}
this is convenient to calculate since, on the left-hand edge,
\[
\partial_{xx} +\nu\partial_{yy} = 0
\]
\subsection{Laplacian finite difference at a corner}
At the bottom-left corner we have, from equation \ref{boundary1},
\[
\partial_{xx} = - \nu \partial_{yy} 
\]
and
\[
\partial_{yy} = - \nu \partial_{xx} 
\]
Since $\nu$ is positive, this implies $\partial_{xx} = 0$ and $\partial_{yy} = 0$ so $\nabla^2 = 0$

At the corner, we cannot differentiate past the corner, so we need to calculate $\partial_{x-yy}$ in a different way:
\[
\partial_{x-yy} = \frac{\partial_{x-}(\partial_{y+} - \partial_{y-})}{\Delta y}
\]
so, from equation \ref{boundarycorner}
\[
\partial_{x-yy} = \frac{\partial_{x-y+}}{\Delta y}
\]
However,  along the left-hand edge
\[
\frac{\partial_{x+} - \partial_{x-}}{\Delta x}=-\nu\partial_{yy}
\]
so
\[
 \partial_{x-}=\Delta x\nu\partial_{yy} + \partial_{x+}
\]
so
\[
\partial_{x-yy} = \frac{\partial_{y+}(\Delta x\nu\partial_{yy} + \partial_{x+})}{\Delta y}
\]
substituting this into equation \ref{xdelta}
\[
\partial_{xx}\nabla^2 = \frac{\Delta y\partial_{x+}\nabla^2 - (1-\nu)\partial_{y+}(\Delta x\nu\partial_{yy} + \partial_{x+})}{\Delta x\Delta y}
\]
by symmetry
\[
\partial_{yy}\nabla^2 = \frac{\Delta x\partial_{y+}\nabla^2 - (1-\nu)\partial_{x+}(\Delta y\nu\partial_{xx} + \partial_{y+})}{\Delta x\Delta y}
\]
so, at the bottom-left corner
\begin{equation}
\nabla^2\nabla^2 = \frac{\Delta x\partial_{y+}\nabla^2 + \Delta y\partial_{x+}\nabla^2 - (1-\nu)(\partial_{y+}(\Delta x\nu\partial_{yy} + \partial_{x+}) + \partial_{x+}(\Delta y\nu\partial_{xx} + \partial_{y+}))}{\Delta x\Delta y}
\end{equation}
simplifying gives
\begin{equation}
\nabla^2\nabla^2 = \frac{\partial_{y+}(\nabla^2- \nu(1-\nu)\partial_{yy})}{\Delta y} + \frac{\partial_{x+}(\nabla^2-\nu(1-\nu)\partial_{xx} )}{\Delta x} - \frac{2(1-\nu)\partial_{x+y+}}{\Delta x\Delta y}
\end{equation}
which is convenient to calculate as, on the corner, $\nabla^2 - \nu(1-\nu)\partial_{yy} = 0$ and $\nabla^2 - \nu(1-\nu)\partial_{xx} = 0$
%%\tableofcontents

%%\appendix

\end{document}

\documentclass[a4paper]{article}
\usepackage{graphics}
%%\usepackage[english,greek]{babel}
\setlength{\parindent}{0mm}
\setlength{\parskip}{1.8mm}
\newtheorem{definition}{Definition}
\newtheorem{theorem}{Theorem}
\newtheorem{proof}{Proof}
\title{Integrating a guitar string over time}
%%\date{$28^{th}$ August, 2007}
\author{Daniel Tang}
%%\linespread{1.3}

\begin{document}
%%\selectlanguage{english}
\maketitle
\section{Third order solution}
Starting with
\[
\frac{\partial^2 y}{\partial t^2} = A\frac{\partial^2y}{\partial x^2} + B \frac{\partial y}{\partial t}
\]
discretise this over space to give
\begin{equation}
\frac{\partial^2 y_i}{\partial t^2} = \frac{A}{\Delta x^2}(y_{i-1} - 2y_i + y_{i+1}) + B \frac{\partial y_i}{\partial t}
\label{equationofmotion}
\end{equation}
Now take the third order Taylor approximation of the position of each point
\[
y_i(t) = y_i(0) + \frac{\partial y_i(0)}{\partial t}t + \frac{1}{2}\frac{\partial^2 y_i(0)}{\partial t^2}t^2 + \frac{1}{6}\frac{\partial^3 y_i(0)}{\partial t^3}t^3
\]
which gives
\[
\frac{\partial y_i(t)}{\partial t} = \frac{\partial y_i(0)}{\partial t} + \frac{\partial^2 y_i(0)}{\partial t^2}t + \frac{1}{2}\frac{\partial^3 y_i(0)}{\partial t^3}t^2
\]
and
\[
\frac{\partial^2 y_i(t)}{\partial t^2} = \frac{\partial^2 y_i(0)}{\partial t^2} + \frac{\partial^3 y_i(0)}{\partial t^3}t
\]
substituting back into equation \ref{equationofmotion} gives us an equation of the form
\[
a\frac{\partial^3 y_{i-1}(0)}{\partial t^3} + b\frac{\partial^3 y_i(0)}{\partial t^3} + a\frac{\partial^3 y_{i+1}(0)}{\partial t^3} = d
\]
where
\[
a = \frac{At^3}{6\Delta x^2}
\]
\[
b = \frac{-2At^3}{6\Delta x^2} + \frac{Bt^2}{2} - t
\]
\[
d = a_i(0) - \left(\frac{A}{\Delta x^2}(y_{i-1,2}(t) - 2y_{i,2}(t) + y_{i+1,2}(t)) + Bv_{i,2}(t)\right)
\]
where $y_{i,2}$ and $v_{i,2}$ are the second order approximations of the position and velocity of the $i^{th}$ point respectively. $d$ can be understood as the change in the diagnosed acceleration between the start and end of a second order timestep. 

So we have a tridiagonal system of equations in the third derivative at each point, which we can solve with the tridiagonal algorithm. If we solve this at time $\Delta t$ (the end of a timestep) then we ensure that position, velocity and acceleration are all continuous between timesteps; i.e. the simulated trajectory of each point is a cubic spline.

%%\tableofcontents

%%\appendix

\end{document}
